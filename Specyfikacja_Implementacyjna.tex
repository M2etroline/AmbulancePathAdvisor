\documentclass{article} 

\usepackage[english]{babel} 
\usepackage{amssymb}
\usepackage{amsmath}
\usepackage{txfonts}
\usepackage{mathdots}
\usepackage[classicReIm]{kpfonts}
\usepackage{graphicx}


\begin{document}


\noindent 

\noindent 

\noindent 

\noindent 

\noindent 

\section{Ambulance Path Advisor -
Specyfikacja implementacyjna}

\noindent 

\noindent 

\noindent 

\noindent 

\noindent 

\noindent 

\noindent 

\noindent 

\noindent \textit{Wydzia{\l}: Elektryczny}

\noindent \textit{Kierunek: Informatyka stosowana}

\noindent \textit{Przedmiot: Algorytmy i struktury danych}

\noindent \textit{}

\noindent \textit{Edvin Suchodolskij}

\noindent \textit{Konrad \v{Z}ilinski}

\noindent \textit{Mateusz Pietrzak}

\noindent 

\eject

\section{ Spis tre\'{s}ci:}

\noindent 

\noindent Opis og\'{o}lny.....................................................................................................................................................3

\noindent Struktura folder\'{o}w.........................................................................................................................................3

\noindent Opis klas.........................................................................................................................................................3

\noindent Opis algorytmu..............................................................................................................................................4

\noindent GUI...................................    .............................................................................................................................4

\noindent Scenariusz dzia{\l}ania programu....................................................................................................................4

\noindent Testowanie.....................................................................................................................................................5

\eject

\noindent 
\section{Opis og\'{o}lny}

\noindent Program s{\l}u\.{z}y dla S{\l}u\.{z}by Ochrony Zdrowia, kt\'{o}rej celem jest pomoc pacjentom dotkni\k{e}tych now\k{a} pandemi\k{a}, kt\'{o}re znajduj\k{a} si\k{e} na terenie kraju. Program wyznacza drog\k{e} dla karetki pogotowia, czyli kieruje karetk\k{a} razem z pacjentem do najbli\.{z}szego o\'{s}rodka medycznego, w celu znalezienia najkr\'{o}tszej drogi do szpitala maj\k{a}cego swobodne {\l}\'{o}\.{z}ko dla pacjenta. W wypadku, gdy w danym szpitalu nie zosta{\l}o si\k{e} wi\k{e}cej swobodnych {\l}\'{o}\.{z}ek, program naprawia pacjenta do najbli\.{z}szego szpitala, w kt\'{o}rym jeszcze dany pacjent nie by{\l}.

\noindent 
\section{Struktura folder\'{o}w}

\noindent \textbf{W projekcie rozr\'{o}\.{z}niamy 3 g{\l}\'{o}wne foldery:}

\noindent {}

\noindent 1) src -- folder, w kt\'{o}rym jest umieszczony ca{\l}y kod licz\k{a}cy.

\noindent 2) test -- folder, w kt\'{o}rym jest umieszczony kod testuj\k{a}cy.

\noindent 3) doc -- folder, w kt\'{o}rym s\k{a} umieszczone specyfikacja implementacyjna i funkcjonalna.

\noindent 
\section{Opis klas}

\noindent W programie jest 7 klas:

\begin{enumerate}
\item  Klasa \textbf{Hospital} -- przechowuje informacje o id, nazwie, po{\l}o\.{z}eniu, {\l}\'{o}\.{z}kachszpitala i wszystkich mo\.{z}liwych bezpo\'{s}rednich drogachwychodz\k{a}cych od tego szpitala do innychobiekt\'{o}w(szpitali lub skrzy\.{z}owa\'{n}). 

\item  Klasa \textbf{Object} -- przechowuje informacje o id, nazwie i powo\.{z}eniuobiekt\'{o}w.

\item  Klasa \textbf{Cross} -- przechowuje informacje o id, po{\l}o\.{z}eniu skrzy\.{z}owa\'{n} i wszystkich mo\.{z}liwych bezpo\'{s}rednich polaczeniach do innych obiekt\'{o}w (szpitali i innych skrzy\.{z}owa\'{n}). 

\item  Klasa \textbf{Patient} -- przechowuje aktualne po{\l}o\.{z}eniepacjenta. 

\item  Klasa \textbf{Map} -- szczytuje dane szpitali, dr\'{o}g oraz objekt\'{o}w, a tak\.{z}e tworzy map\k{e} kraju.

\item  Klasa \textbf{FindHospital} -- znajduje szpital dla pacjenta. 

\item  Klasa \textbf{Main} -- zarz\k{a}dza programem, szczytuje oraz wypisuje dane o pacjentach.
\end{enumerate}

\noindent 
\section{Opis algorytmu}

\noindent W danym projekcie u\.{z}yjemy algorytmu \textbf{\textit{Dijkstry}}. Wybrali\'{s}my go, poniewa\.{z} pomaga w znalezieniu najkr\'{o}tszej drogi do ka\.{z}dego objektu, nie sprawdzaj\k{a}c wszystkich mo\.{z}liwych dr\'{o}g. Algorytm wygl\k{a}da nast\k{e}puj\k{a}co:

\begin{enumerate}

\item Program otrzymuje punkt z kr\'{o}tego rozpoczyna obliczenia.
\item Znajduje najkr\'{o}tsz\k{a} drog\k{e} do s\k{a}siedniego oczka.
\item Przemieszcza si\k{e} do danego oczka, zapisuj\k{a}c jak\k{a} pokona{\l} drog\k{e}, oraz najkr\'{o}tsz\k{a} drog\k{e} do konkretnego oczka.
\item Kontynuje kroki 2-3 do momentu, a\.{z} nie znajdziemy najkr\'{o}tsz\k{a} drog\k{e} do ka\.{z}dego oczka.
\item W wypadku, gdyodleg{\l}o\'{s}\'{c} do zapisanych oczek jest r\'{o}wna, program bierze dowoln\k{a} z wcze\'{s}niej zapisanychdr\'{o}g i wraca, a\.{z} okarze si\k{e} w oczku, do kt\'{o}rego drog\k{e} wybra{\l}.
\item Je\.{z}eli nowa droga do ju\.{z} przeanalizowanego oczka jest d{\l}u\.{z}sza od dotychczas zapisanej, to zostawiamy kr\'{o}tsz\k{a} drog\k{e}.
\end{enumerate}

\section{GUI}

\noindent GUI s{\l}u\.{z}y do wybrania \'{s}cie\.{z}ki obu plik\'{o}w wej\'{s}ciowych oraz pliku wyj\'{s}ciowego. Pliku z danymi mapy szpitali oraz pliku z pacjentami. Dodatkowo, u\.{z}ytkownik b\k{e}dzie m\'{o}g{\l} zaobserwowa\'{c} przebieg dzia{\l}ania programu w postaci animacji wy\'{s}wietlanej w oknie.Cykle animacji uruchamiane b\k{e}d\k{a} po sobie b\k{a}d\'{z} manualnie. wybraniu wszystkich tych element\'{o}w i wci\'{s}ni\k{e}cie przycisku SUBMIT, program stworzy nowe okno, w kt\'{o}rym zostanie wy\'{s}wietlona animacja.

\noindent Przebieg animacji - Karetka b\k{e}dzie si\k{e} przemieszcza{\l}a do miejsc docelowych w trakcie jednego ustalonego cyklu.

\noindent 
\section{Scenariusz dzia{\l}ania programu}

\noindent Po uruchomieniu GUI i klikni\k{e}ciu przycisku SUBMIT program poczyna nast\k{e}puj\k{a}ce kroki:

\begin{enumerate}
\item Sprawdza poprawno\'{s}\'{c} pliku wej\'{s}ciowego.
\item Czyta dane.
\item Tworzy map\k{e} sk{\l}adaj\k{a}c\k{a} si\k{e} z szpitali i objekt\'{o}w.
\item Szuka potencjalnych skrzy\.{z}owa\'{n} dr\'{o}g
\item Sprawdza poprawno\'{s}\'{c} pliku wej\'{s}ciowego 2.
\item Pobiera dane o pacjencie.
\item Znajduje now\k{a} lub wykorzystuje ju\.{z} znan\k{a} drog\k{e} do najbli\.{z}szego szpitala, a\.{z} znajdzie wolne miejsce.
\item Zapisuje pokonan\k{a} drog\k{e} pacjenta do pliku wyj\'{s}ciowego.
\item Powtarza kroki 7-8 dla wszystkich pacjent\'{o}w.
\end{enumerate}

\noindent Po wyczerpaniu pacjent\'{o}w zostaje wy\'{s}wietlone repezentacja graficzna pokonanych dr\'{o}g wszystkich pacjent\'{o}w.

\noindent 
\section{Testowanie}

\noindent Do przetestowania b\k{e}dziemy u\.{z}ywali narz\k{e}dzia JUnit 4.0 w celu znalezienia b{\l}\k{e}d\'{o}w w programie lub udowodnienia, \.{z}e program dzia{\l}a we w{\l}a\'{s}ciwy spos\'{o}b. GUI zostanie przetestowane r\k{e}cznie podczas tworzenia aplikacji. Program wykrywa nast\k{e}puj\k{a}ce b{\l}\k{e}dy:

\begin{enumerate}
\item  B{\l}\k{e}dn\k{a} \'{s}cie\.{z}k\k{e} lub brak pliku. 

\item  Niepoprawn\k{a} ilo\'{s}\'{c} nag{\l}\'{o}wk\'{o}w w pliku wej\'{s}ciowym.

\item  Niepoprawne indeksowanie.

\item  Nadmiar element\'{o}w w poszczeg\'{o}lnych klasach.

\item  Niepoprawny format pojedynczej linii w pliku wej\'{s}ciowym.

\item  Niepoprawny typ elementu w pliku wej\'{s}ciowym.
\end{enumerate}


\end{document}

